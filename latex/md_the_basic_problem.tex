In any message based system you have shared functionality across applications.

The ability to send, receive, parse and process messages, is a technical problem that should be centralised in a simple library.

K\-I\-S\-S\-C\-P\-P provides a framework and an application level communications standard, allowing developers to create message based systems quickly and easily. It provides the major functionality for creating client-\/server applications in C++, while allowing developers to focus on getting business rules implemented.

Take for instance, the two theoretical message based system as described in below\-:


\begin{DoxyEnumerate}
\item {\bfseries Distributed Message based system} 
\item {\bfseries Message Broker based system} 
\end{DoxyEnumerate}

Both these architectures have their own problem sets when dealing with code maintenance and application deployment. Depending on the situation you are facing, you could decide to use either one.

K\-I\-S\-S\-C\-P\-P currently leans more towards solving the distributed message based system. This will only be the case while a message broker using the K\-I\-S\-S\-C\-P\-P library, does not exists. This is currently in planning phase and will be released as a separate project.

\section*{Q\&A}

\subsection*{Why not use Google's protocol buffers?}

I wanted to, really, how awesome would it be to say I use Google's cool stuff \-:) But there is the real world of business needs, a world where\-:


\begin{DoxyEnumerate}
\item J\-S\-O\-N is much better supported across a much wider programming and scripting language base. i.\-e. More people would be able to use J\-S\-O\-N than protocol buffers. Think about it. K\-I\-S\-S\-C\-P\-P based applications will need to interface with arbitrary systems. Using J\-S\-O\-N has a much higher chance of success than protocol buffers ever would.
\item The documentation for protocol buffers, suggest that it's only really for message sizes up to 1 Mb. Seeing as I've worked in environments where the 1\-Mb message size is routinely exceeded, I decided on using something more forgiving.
\item Having to re-\/compile and re-\/deploy components every time an interface changes, causes huge issues for all but the smallest of systems. Yes, interfaces don't change every day, but they change often enough to make the lives of all involved in the development process, a royal pain in the posterior.
\item Protocol buffers destroy any possibility of having templitized messages.
\end{DoxyEnumerate}

\subsection*{Why not use C\-O\-R\-B\-A?}


\begin{DoxyEnumerate}
\item Maintaining I\-D\-L files routinely becomes a mess. In the real-\/world, interfaces change just often enough to make this a problem.
\item Again, having to re-\/compile and re-\/deploy components every time an interface changes, causes problems. See my comments on protocol buffers.
\item C\-O\-R\-B\-A destroys any possibility of having templitized messages.
\end{DoxyEnumerate}

\subsection*{Why not use 0mq?}

0mq lacks one huge feature\-: Persistence! There are other issues with it, but persistence is the big one.

\subsection*{Why not use G\-O?}

G\-O is too young to know wither or not it will stick and become an industry standard. If it does, that would be awesome, but I don't see a future where big corporates are going to re-\/factor towards using G\-O for their server side components. 