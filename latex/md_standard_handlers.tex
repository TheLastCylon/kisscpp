Your K\-I\-S\-S\-C\-P\-P derived applications, will be implementing handlers of it's own. But there are certain things that all applications need. Here is a short list of some of those things\-:


\begin{DoxyItemize}
\item keeping and retrieving of statistics
\item keeping, retrieving and clearing application error states.
\item The ability to adjust an applications log level, without restarting the application.
\item The ability to see a list of handlers that an application implements.
\end{DoxyItemize}

K\-I\-S\-S\-C\-P\-P aims to provide standardised mechanisms for all of the above, and more.

Here is a list of the standard handlers, currently implemented by K\-I\-S\-S\-C\-P\-P.

{\itshape The {\bfseries kch} prefix, is shorthand for (K)iss(\-C)pp (H)andler. To prevent conflict between K\-I\-S\-S\-C\-P\-P specific handlers and your own, do not prefix any of your custom handler identifiers with {\bfseries kch}.}

\begin{TabularC}{2}
\hline
\rowcolor{lightgray}{\bf {\bfseries Handler Id} }&{\bf {\bfseries Purpose}  }\\\cline{1-2}
kch-\/loglevel &Adjust the logging to specified levels \\\cline{1-2}
kch-\/handlers &shows a list of the request handlers for this application \\\cline{1-2}
kch-\/status &retrieves the application error states \\\cline{1-2}
kch-\/statistics &retrieves the application statistics \\\cline{1-2}
\end{TabularC}
In order to ease the introduction to this here, we'll start with discussing the adjustment of log levels.

\subsection*{Adjusting log level on the fly.}

\begin{TabularC}{2}
\hline
\rowcolor{lightgray}\PBS\centering {\bf {\bfseries Handler} }&\PBS\centering {\bf {\bfseries Purpose}  }\\\cline{1-2}
\PBS\centering kch-\/loglevel &\PBS\centering Used to set the logging level to new values, while an application is running. \\\cline{1-2}
\end{TabularC}
\subsubsection*{Parameters}

\begin{TabularC}{3}
\hline
\rowcolor{lightgray}{\bf {\bfseries Parameter name} }&\PBS\centering {\bf {\bfseries Mandatory?} }&{\bf {\bfseries possible values}  }\\\cline{1-3}
type &\PBS\centering Yes &\char`\"{}debug\char`\"{}, \char`\"{}info\char`\"{}, \char`\"{}error\char`\"{} \\\cline{1-3}
severity &\PBS\centering Yes &\char`\"{}low\char`\"{}, \char`\"{}normal\char`\"{}, \char`\"{}high\char`\"{} \\\cline{1-3}
\end{TabularC}
\subsubsection*{Examples}


\begin{DoxyItemize}
\item The example blow, will cause a K\-I\-S\-S\-C\-P\-P application to start logging everything at all log levels. For more detail on logging and log levels, please see the section on logging. 
\begin{DoxyCode}
\{\textcolor{stringliteral}{"kcm-cmd"}:\textcolor{stringliteral}{"kch\_loglevel"},\textcolor{stringliteral}{"type"}:\textcolor{stringliteral}{"debug"},\textcolor{stringliteral}{"severity"}:\textcolor{stringliteral}{"low"},\textcolor{stringliteral}{"kcm-client"}:\{\textcolor{stringliteral}{"id"}:\textcolor{stringliteral}{"foo"},\textcolor{stringliteral}{"instance"}:\textcolor{stringliteral}{"1"}\}\}
\end{DoxyCode}

\item And in the second example bellow, logging will be restricted to logs of type {\itshape info} or {\itshape error} with a severity of at least \char`\"{}normal\char`\"{}. 
\begin{DoxyCode}
\{\textcolor{stringliteral}{"kcm-cmd"}:\textcolor{stringliteral}{"kch\_loglevel"},\textcolor{stringliteral}{"type"}:\textcolor{stringliteral}{"info"},\textcolor{stringliteral}{"severity"}:\textcolor{stringliteral}{"normal"},\textcolor{stringliteral}{"kcm-client"}:\{\textcolor{stringliteral}{"id"}:\textcolor{stringliteral}{"foo"},\textcolor{stringliteral}{"instance"}:\textcolor{stringliteral}{"1"}\}\}
\end{DoxyCode}

\end{DoxyItemize}

Logging and it's implementation in K\-I\-S\-S\-C\-P\-P is discussed here.

This is a discussion on how to change the type and severity of logs that are written by an application, while the application is running.

The standard handler for this is called

\subsection*{Format}

After much investigation and debate, J\-S\-O\-N was yet again selected as the preferred format. In the case of configuration files, we found that I\-N\-I format lacks the ability for nested records and X\-M\-L quickly becomes an unreadable mess. The only down side we have identified with the use of J\-S\-O\-N as a configuration file format, was it's lack of native support for comments. This is a shortcoming we are more than willing to live with.

\subsection*{Reserved identifiers}

The following are reserved identifiers, for use in configuration files\-: Note\-: The prefix of {\bfseries kcc} is shorthand for (K)iss(\-C)pp (C)onfiguration.

{\itshape The {\bfseries kcc} prefix, is shorthand for (K)iss(\-C)pp (C)onfiguration. To prevent conflict between K\-I\-S\-S\-C\-P\-P specific configuration and your own application data, do not prefix any of your custom configuration variables with {\bfseries kcc}.}

\begin{TabularC}{2}
\hline
\rowcolor{lightgray}{\bf {\bfseries Identifier} }&{\bf {\bfseries Description}  }\\\cline{1-2}
kcc-\/server.\-address &the hostname or ip address of the server. \\\cline{1-2}
kcc-\/server.\-port &the port of the server. \\\cline{1-2}
kcc-\/stats.\-gather-\/period &Seconds between gathering statistics for historic purposes. \\\cline{1-2}
kcc-\/stats.\-history-\/length &Number of historic stats gatherings to keep. \\\cline{1-2}
kcc-\/log-\/level.\-type &The default type limitation on logs. \\\cline{1-2}
kcc-\/log-\/level.\-severity &The default severity limitation on logs. \\\cline{1-2}
kcc-\/white-\/list &A root node containing data regarding white list communications. More detail available \href{md_white_listed_communications.html}{\tt here}. \\\cline{1-2}
\end{TabularC}
\subsection*{Configuration file naming standard}

K\-I\-S\-S\-C\-P\-P applications will have configuration files named as follows\-:


\begin{DoxyCode}
<application-\textcolor{keywordtype}{id}>.<application-instance-\textcolor{keywordtype}{id} or \textcolor{stringliteral}{"common"}>.kcppcfg
\end{DoxyCode}


i.\-e. An application with the id \char`\"{}foo\char`\"{}, being executed with an instance identifier of \char`\"{}bar\char`\"{} will look for a configuration file named\-:


\begin{DoxyCode}
foo.bar.kcppcfg
\end{DoxyCode}


This file must contain configuration data that is specific to the {\bfseries foo} application running as instance {\bfseries bar}.

There is however, a mechanism for allowing shared configuration to exist in a single file. This is known as the common-\/configuration file. This file will have the instance id portion of the file name, replaced by the string \char`\"{}common\char`\"{}.

i.\-e. An application with the id {\bfseries foo}, being executed with an instance identifier of {\bfseries bar} will look for shared configuration details in a file named.


\begin{DoxyCode}
foo.common.kcppcfg
\end{DoxyCode}


\subsection*{Configuration file location}

Due to the fact that K\-I\-S\-S\-C\-P\-P is a library meant for large systems consisting of interdependent processes, we did have to standardize on most aspects of the configuration files. The control we can give you, is based around a holistic approach, and the fact that the developer is not the only one that needs to be considered here. System administrators and support personnel also need to have a sane standard to work with.

As such, the following decision were made\-:


\begin{DoxyEnumerate}
\item Only the root path for configuration files, will be under user control. i.\-e. You can specify nothing more than the path under which K\-I\-S\-S\-C\-P\-P will search for the following\-:
\begin{DoxyItemize}
\item A sub-\/directory with a name matching the application-\/id.
\item Files contained in that sub-\/directory, that match the file naming standard discussed above.
\end{DoxyItemize}
\item A mechanism for environment separation will be available. i.\-e. You have the ability to keep environment specific files separated from each other. This mechanism is provided through the {\bfseries K\-C\-P\-P\-\_\-\-E\-X\-E\-C\-\_\-\-E\-N\-V} environment variable
\end{DoxyEnumerate}

The configuration file path, is constructed in one of three ways\-:


\begin{DoxyItemize}
\item Where {\bfseries K\-C\-P\-P\-\_\-\-C\-F\-G\-\_\-\-R\-O\-O\-T} and {\bfseries K\-C\-P\-P\-\_\-\-E\-X\-E\-C\-\_\-\-E\-N\-V} are used\-: \$\-K\-C\-P\-P\-\_\-\-C\-F\-G\-\_\-\-R\-O\-O\-T/$<$application-\/id$>$/\$\-K\-C\-P\-P\-\_\-\-E\-X\-E\-C\-\_\-\-E\-N\-V/$<$application-\/configuration-\/file-\/name$>$
\item Where only {\bfseries K\-C\-P\-P\-\_\-\-C\-F\-G\-\_\-\-R\-O\-O\-T} is used\-: \$\-K\-C\-P\-P\-\_\-\-C\-F\-G\-\_\-\-R\-O\-O\-T/$<$application-\/id$>$/$<$application-\/configuration-\/file-\/name$>$
\item Developers can override any setting of {\bfseries K\-C\-P\-P\-\_\-\-C\-F\-G\-\_\-\-R\-O\-O\-T} by specifying the path at the time of instantiating the Server class.
\end{DoxyItemize}

That means, when you have a K\-I\-S\-S\-C\-P\-P application with the name {\bfseries foo} and you start instance {\bfseries bar} of it, while having {\bfseries K\-C\-P\-P\-\_\-\-C\-F\-G\-\_\-\-R\-O\-O\-T} set to {\bfseries /put/configurations/here}

Your application will look in\-:
\begin{DoxyItemize}
\item /put/configurations/here/foo
\item for a files named
\begin{DoxyItemize}
\item foo.\-common.\-kcppcfg and
\item foo.\-bar.\-kcppcfg
\end{DoxyItemize}
\end{DoxyItemize}

Meaning that the full path to these files will be\-:
\begin{DoxyItemize}
\item /put/configurations/here/foo/foo.common.\-kcppcfg and
\item /put/configurations/here/foo/foo.bar.\-kcppcfg
\end{DoxyItemize}

Now, with all of that in play, you can still set the {\bfseries K\-C\-P\-P\-\_\-\-E\-X\-E\-C\-\_\-\-E\-N\-V} environment variable. This comes in handy when you have differing configurations for your test, quality control and production environments.

Thus, with all the above and {\bfseries K\-C\-P\-P\-\_\-\-E\-X\-E\-C\-\_\-\-E\-N\-V} set too\-: {\bfseries live}

Your application will look in\-:
\begin{DoxyItemize}
\item /put/configurations/here/foo/live
\item for files named
\begin{DoxyItemize}
\item foo.\-common.\-kcppcfg and
\item foo.\-bar.\-kcppcfg
\end{DoxyItemize}
\end{DoxyItemize}

Meaning that the full path to these files will be\-:
\begin{DoxyItemize}
\item /put/configurations/here/foo/live/foo.common.\-kcppcfg and
\item /put/configurations/here/foo/live/foo.bar.\-kcppcfg
\end{DoxyItemize}

\subsection*{Example.}

Ok, let's say you have an application called {\bfseries ninjarules} and you need to run three instances of that application on the same machine.

To keep things sane, you set {\bfseries K\-C\-P\-P\-\_\-\-C\-F\-G\-\_\-\-R\-O\-O\-T} too {\bfseries /my/central/config}. You also know that your test and your live environments will have different configurations so you plan on setting {\bfseries K\-C\-P\-P\-\_\-\-E\-X\-E\-C\-\_\-\-E\-N\-V} to {\bfseries live} and {\bfseries test} respectively, so that you can avoid having to maintain both sets of configurations in one file.

You further decide, although you can use any arbitrary string as an instance identifier, that you'll stick to the simple numbers 1, 2 and 3, for your various instances.

Your application will therefore look in these directories\-:
\begin{DoxyItemize}
\item In the {\bfseries live} environment\-: /my/central/config/ninjarules/live
\item In the {\bfseries test} environment\-: /my/central/config/ninjarules/test
\end{DoxyItemize}

For files named\-:
\begin{DoxyItemize}
\item ninjarules.\-common.\-kcppcfg,
\item ninjarules.\-1.\-kcppcfg,
\item ninjarules.\-2.\-kcppcfg and
\item ninjarules.\-3.\-kcppcfg
\end{DoxyItemize}

There are some things that will be common for all instances of {\bfseries ninjarules}, so you'll want to keep them in the common configuration. Here's an example of that\-: 
\begin{DoxyCode}
\{(ninjarules.common.json)\}
\{
  \textcolor{stringliteral}{"kcc-stats"}     : \{
    \textcolor{stringliteral}{"gather-period"}  : \textcolor{stringliteral}{"300"},
    \textcolor{stringliteral}{"history-length"} : \textcolor{stringliteral}{"12"}
  \},

  \textcolor{stringliteral}{"kcc-log-level"} : \{
    \textcolor{stringliteral}{"type"}           : \textcolor{stringliteral}{"info"},
    \textcolor{stringliteral}{"severity"}       : \textcolor{stringliteral}{"normal"}
  \}
\}
\end{DoxyCode}


And then of course, there will be the instance level configuration files\-: 
\begin{DoxyCode}
\{(ninjarules.1.kcppcfg)\}
\{
  \textcolor{stringliteral}{"kcc-server"}    : \{
    \textcolor{stringliteral}{"address"}        : \textcolor{stringliteral}{"locahost"},
    \textcolor{stringliteral}{"port"}           : \textcolor{stringliteral}{"9001"}
  \}
\}
\end{DoxyCode}
 
\begin{DoxyCode}
\{(ninjarules.2.kcppcfg)\}
\{
  \textcolor{stringliteral}{"kcc-server"}    : \{
    \textcolor{stringliteral}{"address"}        : \textcolor{stringliteral}{"locahost"},
    \textcolor{stringliteral}{"port"}           : \textcolor{stringliteral}{"9002"}
  \}
\}
\end{DoxyCode}
 
\begin{DoxyCode}
\{(ninjarules.3.kcppcfg)\}
\{
  \textcolor{stringliteral}{"kcc-server"}    : \{
    \textcolor{stringliteral}{"address"}        : \textcolor{stringliteral}{"locahost"},
    \textcolor{stringliteral}{"port"}           : \textcolor{stringliteral}{"9003"}
  \}
\}
\end{DoxyCode}
 