\subsection*{Format}

After much investigation and debate, J\-S\-O\-N was yet again selected as the preferred format. In the case of configuration files, we found that I\-N\-I format lacks the ability for nested records and X\-M\-L quickly becomes an unreadable mess. The only down side we have identified with the use of J\-S\-O\-N as a configuration file format, was it's lack of native support for comments. This is a shortcoming we are more than willing to live with. If you need comments in your configuration files, you actually need way more than just a configuration file.

\subsection*{Reserved identifiers}

The following are reserved identifiers, for use in configuration files\-: Note\-: The prefix of {\bfseries kcc} is shorthand for (K)iss(\-C)pp (C)onfiguration.

\begin{TabularC}{2}
\hline
\rowcolor{lightgray}{\bf $\ast$$\ast$\-Identifier$\ast$$\ast$ }&{\bf {\bfseries Description}  }\\\cline{1-2}
kcc-\/server.\-address &the hostname or ip address of the server. \\\cline{1-2}
kcc-\/server.\-port &the port of the server. \\\cline{1-2}
kcc-\/stats.\-gather-\/period &Seconds between gathering statistics for historic purposes. \\\cline{1-2}
kcc-\/stats.\-history-\/length &Number of historic stats gatherings to keep. \\\cline{1-2}
kcc-\/log-\/level.\-type &The default type limitation on logs. \\\cline{1-2}
kcc-\/log-\/level.\-severity &The default severity limitation on logs. \\\cline{1-2}
kcc-\/white-\/list &A root node containing data regarding whilte list communications. More detail agvailble {\bfseries T\-O\-D\-O\-: Add link to more detail} \\\cline{1-2}
\end{TabularC}
\subsection*{Configuration file naming standard}

K\-I\-S\-S\-C\-P\-P applications will have configuration files named as follows\-:


\begin{DoxyCode}
<application-\textcolor{keywordtype}{id}>.<application-instance-\textcolor{keywordtype}{id} or \textcolor{stringliteral}{"common"}>.kcppcfg
\end{DoxyCode}


i.\-e. An application with the id \char`\"{}foo\char`\"{}, being executed with an instance identifier of \char`\"{}bar\char`\"{} will look for a configuration file named\-:


\begin{DoxyCode}
foo.bar.kcppcfg
\end{DoxyCode}


This file must contain configuration data that is specific to the {\bfseries foo} application running as instance {\bfseries bar}.

There is however, a mechanism for allowing shared configuration to exist in a single file. This is known as the common-\/configuration file. This file will have the instance id portion of the file name, replaced by the string \char`\"{}common\char`\"{}.

i.\-e. An application with the id {\bfseries foo}, being executed with an instance identifier of {\bfseries bar} will look for shared configuration details in a file named.


\begin{DoxyCode}
foo.common.kcppcfg
\end{DoxyCode}


\subsection*{Configuration file location}

Due to the fact that K\-I\-S\-S\-C\-P\-P is a library meant for large systems consisting of interdependent processes, we did have to standardize on most aspects of the configuration files. The control we can give you, is based around a holistic approach, and the fact that the developer is not the only one that needs to be considered here. System administrators and support personnel also need to have a sane standard to work with.

K\-I\-S\-S\-C\-P\-P applications will look in $\ast$$\ast$/etc/kisscpp$\ast$$\ast$ for configuration files, by default. This can be overriden in three ways\-:


\begin{DoxyEnumerate}
\item Programatically, through the explicit specification of a root path, at the point where you instantiate the Server class.
\item Specifying the root path in the environment variable {\bfseries K\-C\-P\-P\-\_\-\-C\-F\-G\-\_\-\-R\-O\-O\-T}, while leaving the environment variable {\bfseries K\-C\-P\-P\-\_\-\-A\-L\-L\-\_\-\-R\-O\-O\-T} un-\/set.
\item Specifying the root path as a combination of $\ast$$\ast$\-K\-C\-P\-P\-\_\-\-C\-F\-G\-\_\-$\ast$$\ast$ and {\bfseries K\-C\-P\-P\-\_\-\-A\-L\-L\-\_\-\-R\-O\-O\-T}.
\end{DoxyEnumerate}

In the case of (1) above, you have full controll of the root path, for any one process. In (2) and (3), you are controlling the root path for all K\-I\-S\-S\-C\-P\-P derived applications.


\begin{DoxyEnumerate}
\item The first part of
\end{DoxyEnumerate}


\begin{DoxyCode}
\{(example.common.kcppcfg)\}
\{
  \textcolor{stringliteral}{"kcc-server"}    : \{
    \textcolor{stringliteral}{"address"}        : \textcolor{stringliteral}{"locahost"},
    \textcolor{stringliteral}{"port"}           : \textcolor{stringliteral}{"9000"}
  \},

  \textcolor{stringliteral}{"kcc-stats"}     : \{
    \textcolor{stringliteral}{"gather-period"}  : \textcolor{stringliteral}{"300"},
    \textcolor{stringliteral}{"history-length"} : \textcolor{stringliteral}{"12"}
  \},

  \textcolor{stringliteral}{"kcc-log-level"} : \{
    \textcolor{stringliteral}{"type"}           : \textcolor{stringliteral}{"info"},
    \textcolor{stringliteral}{"severity"}       : \textcolor{stringliteral}{"normal"}
  \}
\}
\end{DoxyCode}


\subsection*{File Location}


\begin{DoxyEnumerate}
\item For networking boost\-::asio is used.

\href{md_inter_process_communication.html}{\tt Inter Process Communication}
\item For networking boost\-::asio is used. 
\end{DoxyEnumerate}