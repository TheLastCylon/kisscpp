Well, there are actually several reasons, but the most important ones are listed here.

\section*{Bad code}

I used to work in an environment where I had to maintain source code containing C functions, spanning 2000 lines. One or two such functions would not really have bothered me, as I would simply conclude that the code was done under pressure, and a re-\/factor was in order. The problem here though, was the 37 near duplicate functions each spanning said 2000 lines of code.

At one point, I was convinced that this kind of code, was a result of incompitant developers in that environment. I was partially correct, in that there were developers that were incompitent. However, this was a reflection of management and the prevailing employment strategies, as well as a non-\/existant internal communication structure.

It was the kind of environment, where non-\/technical people were placed in senior technical positions, resulting in decisions around tooling being made by people who never use the tools.

Sadly, I'm not exagirating. If I were, I'd probably still be working in that environment.

In short though, it was the typical Waterfall development, perpetuated by a poor implimentation of the Rational Unified Process, where hamstrung project-\/managers had no choice but to pester hamstrung developers into hacking a 15 year old, poorly designed system, so as to provide the appearance of valuable solutions to the business.

K\-I\-S\-S\-C\-P\-P is the culmination of frustration and experiance, into a library that will make it possible for developers at lartge corporates, to deliver re-\/factored components, at a fraction of the cost of maintaining old ones.

Yes, the system will have to be re-\/designed to make use of K\-I\-S\-S\-C\-P\-P, but the cost would be rediculously low compared to maintaining a system like the one I had to deal with.

\section*{Development speed}

Over the years I've had to listen to the \char`\"{}\-My language is better than yours\char`\"{} debate, one too many times. The more I litened to these debates, the more I realized that there are several types of Software Developers.


\begin{DoxyEnumerate}
\item {\bfseries Those that don't know.} \begin{quotation}
These are the ones, that are just starting out. They are totaly new to the industry and can be forgiven almost anything.

\end{quotation}

\item {\bfseries Those that think they know.} \begin{quotation}
These are the ones that will blindly stick to a single programming language/ tool/methedology/paradigm etc. They will happily argue that a business will only need the tool they happen to be proficient in. They are also the kind of developer that will argue, that Delphi is a programming language and P\-H\-P is fast enough to be used as a tool for creating a database cache service.

\end{quotation}

\item {\bfseries Those that know.} \begin{quotation}
These are the ones that actually get qualified in multiple programming languages, using diverse paradigms. They are the ones that will convince a business of the need for a multitude of tools and skills to manage it's own software. They will prescribe P\-H\-P for the web requirements, Java for business components, C++ for the componets that feel the need for speed, C for those that feel the need for warp speed, assembler for trans-\/warp speeds and machine code for instantanious teleportation.

\end{quotation}

\end{DoxyEnumerate}

There is a particular kind of the \char`\"{}\-Think they know\char`\"{} developer, that I aim to address, using the K\-I\-S\-S\-C\-P\-P library. They are the ones who argue, that getting something done in C++ takes too much development effort. With the release of K\-I\-S\-S\-C\-P\-P and the various applications that use it, I aim to demonstrate just how fast one can develop in C++, when you are willing to standardise your development and/or design.

The only advantage a programming language like Java has over C++, is that J\-A\-V\-A developers have been forced to standardise their solutions.

K\-I\-S\-S\-C\-P\-P is a step in that direction, for C++ developers. 